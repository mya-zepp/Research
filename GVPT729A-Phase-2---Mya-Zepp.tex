% Options for packages loaded elsewhere
\PassOptionsToPackage{unicode}{hyperref}
\PassOptionsToPackage{hyphens}{url}
\PassOptionsToPackage{dvipsnames,svgnames,x11names}{xcolor}
%
\documentclass[
  12pt,
  letterpaper,
  DIV=11,
  numbers=noendperiod]{scrartcl}

\usepackage{amsmath,amssymb}
\usepackage{setspace}
\usepackage{iftex}
\ifPDFTeX
  \usepackage[T1]{fontenc}
  \usepackage[utf8]{inputenc}
  \usepackage{textcomp} % provide euro and other symbols
\else % if luatex or xetex
  \usepackage{unicode-math}
  \defaultfontfeatures{Scale=MatchLowercase}
  \defaultfontfeatures[\rmfamily]{Ligatures=TeX,Scale=1}
\fi
\usepackage{lmodern}
\ifPDFTeX\else  
    % xetex/luatex font selection
    \setmainfont[]{Times New Roman}
\fi
% Use upquote if available, for straight quotes in verbatim environments
\IfFileExists{upquote.sty}{\usepackage{upquote}}{}
\IfFileExists{microtype.sty}{% use microtype if available
  \usepackage[]{microtype}
  \UseMicrotypeSet[protrusion]{basicmath} % disable protrusion for tt fonts
}{}
\makeatletter
\@ifundefined{KOMAClassName}{% if non-KOMA class
  \IfFileExists{parskip.sty}{%
    \usepackage{parskip}
  }{% else
    \setlength{\parindent}{0pt}
    \setlength{\parskip}{6pt plus 2pt minus 1pt}}
}{% if KOMA class
  \KOMAoptions{parskip=half}}
\makeatother
\usepackage{xcolor}
\usepackage[lmargin=1in,rmargin=1in,tmargin=1in,bmargin=1in]{geometry}
\setlength{\emergencystretch}{3em} % prevent overfull lines
\setcounter{secnumdepth}{-\maxdimen} % remove section numbering
% Make \paragraph and \subparagraph free-standing
\makeatletter
\ifx\paragraph\undefined\else
  \let\oldparagraph\paragraph
  \renewcommand{\paragraph}{
    \@ifstar
      \xxxParagraphStar
      \xxxParagraphNoStar
  }
  \newcommand{\xxxParagraphStar}[1]{\oldparagraph*{#1}\mbox{}}
  \newcommand{\xxxParagraphNoStar}[1]{\oldparagraph{#1}\mbox{}}
\fi
\ifx\subparagraph\undefined\else
  \let\oldsubparagraph\subparagraph
  \renewcommand{\subparagraph}{
    \@ifstar
      \xxxSubParagraphStar
      \xxxSubParagraphNoStar
  }
  \newcommand{\xxxSubParagraphStar}[1]{\oldsubparagraph*{#1}\mbox{}}
  \newcommand{\xxxSubParagraphNoStar}[1]{\oldsubparagraph{#1}\mbox{}}
\fi
\makeatother


\providecommand{\tightlist}{%
  \setlength{\itemsep}{0pt}\setlength{\parskip}{0pt}}\usepackage{longtable,booktabs,array}
\usepackage{calc} % for calculating minipage widths
% Correct order of tables after \paragraph or \subparagraph
\usepackage{etoolbox}
\makeatletter
\patchcmd\longtable{\par}{\if@noskipsec\mbox{}\fi\par}{}{}
\makeatother
% Allow footnotes in longtable head/foot
\IfFileExists{footnotehyper.sty}{\usepackage{footnotehyper}}{\usepackage{footnote}}
\makesavenoteenv{longtable}
\usepackage{graphicx}
\makeatletter
\newsavebox\pandoc@box
\newcommand*\pandocbounded[1]{% scales image to fit in text height/width
  \sbox\pandoc@box{#1}%
  \Gscale@div\@tempa{\textheight}{\dimexpr\ht\pandoc@box+\dp\pandoc@box\relax}%
  \Gscale@div\@tempb{\linewidth}{\wd\pandoc@box}%
  \ifdim\@tempb\p@<\@tempa\p@\let\@tempa\@tempb\fi% select the smaller of both
  \ifdim\@tempa\p@<\p@\scalebox{\@tempa}{\usebox\pandoc@box}%
  \else\usebox{\pandoc@box}%
  \fi%
}
% Set default figure placement to htbp
\def\fps@figure{htbp}
\makeatother

\usepackage{float}
\usepackage{tabularray}
\usepackage[normalem]{ulem}
\usepackage{graphicx}
\usepackage{rotating}
\UseTblrLibrary{booktabs}
\UseTblrLibrary{siunitx}
\NewTableCommand{\tinytableDefineColor}[3]{\definecolor{#1}{#2}{#3}}
\newcommand{\tinytableTabularrayUnderline}[1]{\underline{#1}}
\newcommand{\tinytableTabularrayStrikeout}[1]{\sout{#1}}
% Required by modelsummary/tabularray
\usepackage{tabularray}
\usepackage{float}
\usepackage{graphicx}
\usepackage{codehigh}
\usepackage[normalem]{ulem}
\UseTblrLibrary{booktabs}
\UseTblrLibrary{siunitx}
\KOMAoption{captions}{tableheading}
\makeatletter
\@ifpackageloaded{caption}{}{\usepackage{caption}}
\AtBeginDocument{%
\ifdefined\contentsname
  \renewcommand*\contentsname{Table of contents}
\else
  \newcommand\contentsname{Table of contents}
\fi
\ifdefined\listfigurename
  \renewcommand*\listfigurename{List of Figures}
\else
  \newcommand\listfigurename{List of Figures}
\fi
\ifdefined\listtablename
  \renewcommand*\listtablename{List of Tables}
\else
  \newcommand\listtablename{List of Tables}
\fi
\ifdefined\figurename
  \renewcommand*\figurename{Figure}
\else
  \newcommand\figurename{Figure}
\fi
\ifdefined\tablename
  \renewcommand*\tablename{Table}
\else
  \newcommand\tablename{Table}
\fi
}
\@ifpackageloaded{float}{}{\usepackage{float}}
\floatstyle{ruled}
\@ifundefined{c@chapter}{\newfloat{codelisting}{h}{lop}}{\newfloat{codelisting}{h}{lop}[chapter]}
\floatname{codelisting}{Listing}
\newcommand*\listoflistings{\listof{codelisting}{List of Listings}}
\makeatother
\makeatletter
\makeatother
\makeatletter
\@ifpackageloaded{caption}{}{\usepackage{caption}}
\@ifpackageloaded{subcaption}{}{\usepackage{subcaption}}
\makeatother

\usepackage{bookmark}

\IfFileExists{xurl.sty}{\usepackage{xurl}}{} % add URL line breaks if available
\urlstyle{same} % disable monospaced font for URLs
\hypersetup{
  colorlinks=true,
  linkcolor={blue},
  filecolor={Maroon},
  citecolor={Blue},
  urlcolor={Blue},
  pdfcreator={LaTeX via pandoc}}


\author{}
\date{}

\begin{document}


\setstretch{2}
\begin{center}

GVPT 729A: Phase 2 Preliminary Results Assignment

Mya Zepp

October 27, 2025

\end{center}

\pagebreak

In 2020, the Marshall Project and Slate partnered to gather information
on the politics of people behind bars. They conducted a voluntary survey
of 8,000 incarcerated individuals across six states. The majority of
respondents were white men in Kansas and Arkansas, and thus not
representative of the overall prison population. As such, any
conclusions drawn from the data are only truly generalizable to white
incarcerated men in Republican states. The methods used in conducting
this survey are another area of interest. The surveyors had no direct
contact with prisons or imprisoned people; rather, they conducted their
research through a tablet company that provides tablets to incarcerated
people. The survey was given to this company and then sent to the
tablets assigned to individual people.

Using this data, I explore the answer provided to the question ``What
impact has incarceration had on your motivation to vote?'' to which
respondents answered, `increased my motivation to vote', `slightly
increased my motivation to vote', `decreased my motivation to vote',
`slightly decreased my motivation to vote,' or `no impact.' For the
purposes of this analysis, I excluded `no impact' (n = 996). I compare
the results of this question to the length of time the respondents have
spent incarcerated, which is coded as `10 years or less' and `more than
10 years'

Previous research in the field of incarceration and politics has had
mixed opinions on the impact of the prison industrial complex on voting.
One study indicated that any interaction with the carceral state, even a
traffic ticket, can lead to decreased voter turnout (Weaver and Lerman,
2010) while another suggests that incarceration causes no real change in
civic participation (Gerber et al., 2017) and a minority of research
argues, as I will, that some people may in fact be mobilized by
interactions with the carceral state (Walker, 2020). The data I plan to
explore examines the impact of incarceration on `motivation,' not
participation. As such, we cannot make inferences on how these same
respondents will participate, but I hypothesize, similar to Hannah
Walker, that spending more than 10 years in prison will positively
impact motivation to vote (\(H_1\)). In the analysis that follows I
examine these variables through a cross tabulation and through an OLS
regression model. In future analysis I plan to use binary logit to
further explore the relationship and I plan to include controls for
race, party identification, age, and having voted in the past.

Preliminary analysis of the variables demonstrates a relationship that
supports my hypothesis. The cross-tabulation of responses shows that
among respondents who have spent 10 years or less incarcerated, 45.8\%
stated their motivation to vote increased, and respondents who have
spent more than 10 years incarcerated had similar results, with 58.8\%
stating their motivation to vote increased. Length in facility
demonstrates that spending 10 years or more in prison is associated with
an increase in motivation to vote compared to those who have spent less
than 10 years in prison. This result is in line with the argument that
``a pivotal factor connecting experiences with the criminal justice
system to political mobilization is a sense of systemic injustice.''
(Walker, 2020, p.~6) This association could be a result of more time
spent considering the implications of the prison-industrial complex,
resulting in a stronger desire to participate politically. It is also
possible that losing the right to vote makes one have a stronger desire
to vote once they get that right back.

The data suggests that motivation to vote may be high among incarcerated
people, demonstrating that the conventional wisdom suggesting that
previously incarcerated people are not participating in politics due to
a lack of information may very well be true because based on this data,
it is not due to a lack of motivation.

\subsection{Appendix}\label{appendix}

\(H_0\): Spending 10 years or more in prison will negatively impact or
not impact motivation to vote

\(H_1\): Spending 10 years or more in prison will positively impact
motivation to vote

\paragraph{Summary: Incarceration Impact on Motivation to
Vote}\label{summary-incarceration-impact-on-motivation-to-vote}

\begin{verbatim}
   Min. 1st Qu.  Median    Mean 3rd Qu.    Max.    NA's 
 0.0000  1.0000  1.0000  0.8113  1.0000  1.0000     996 
\end{verbatim}

\paragraph{Summary: Time Spent in
Prison}\label{summary-time-spent-in-prison}

\begin{verbatim}
   Min. 1st Qu.  Median    Mean 3rd Qu.    Max. 
 0.0000  0.0000  0.0000  0.1369  0.0000  1.0000 
\end{verbatim}

\begin{table}
\centering
\begin{talltblr}[         %% tabularray outer open
caption={Time Incarcerated Regressed on Voting Motivation},
note{}={+ p \num{< 0.1}, * p \num{< 0.05}, ** p \num{< 0.01}, *** p \num{< 0.001}},
]                     %% tabularray outer close
{                     %% tabularray inner open
colspec={Q[]Q[]},
column{2}={}{halign=c,},
column{1}={}{halign=l,},
hline{4}={1,2}{solid, black, 0.05em},
}                     %% tabularray inner close
\toprule
& (1) \\ \midrule %% TinyTableHeader
Length in Facility & 0.083** \\
& (0.029) \\
Num.Obs. & 1415 \\
R2 & 0.006 \\
\bottomrule
\end{talltblr}
\end{table}

\begin{table}
\centering
\begin{talltblr}[         %% tabularray outer open
caption={Time Incarcerated and Voting Motivation},
]                     %% tabularray outer close
{                     %% tabularray inner open
colspec={Q[]Q[]Q[]Q[]Q[]},
column{1,2}={}{halign=l,},
column{3,4,5}={}{halign=r,},
}                     %% tabularray inner close
\toprule
Time incarcerated &   & Decreased motivation/No change & Increased motivation & All \\ \midrule %% TinyTableHeader
10 years or less & N & 241 & 954 & 2081 \\
& \% row & 11.6 & 45.8 & 100.0 \\
More than 10 years & N & 26 & 194 & 330 \\
& \% row & 7.9 & 58.8 & 100.0 \\
All & N & 267 & 1148 & 2411 \\
& \% row & 11.1 & 47.6 & 100.0 \\
\bottomrule
\end{talltblr}
\end{table}




\end{document}
